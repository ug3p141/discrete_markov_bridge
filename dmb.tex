\documentclass{beamer}
\usetheme{Madrid}
\usecolortheme{seahorse}
\usepackage{amsmath,amssymb}
\usepackage{hyperref}
\title{Discrete Markov Bridge:\\ Bridging Variational Inference and Discrete Diffusion}
\author{Hengli Li, Yuxuan Wang, Song-Chun Zhu, Ying Nian Wu, Zilong Zheng}
\institute{Peking University, Beijing Institute for General AI, UCLA}
\date{\today}

\begin{document}

\begin{frame}
  \titlepage
\end{frame}

\begin{frame}{Outline}
  \tableofcontents
\end{frame}

\section{Motivation and Background}

\begin{frame}{Motivation}
  \textbf{Generative modeling} seeks to estimate an underlying data distribution $\mu$ and generate new samples from it.\\[1em]
  \textbf{Diffusion models} have shown strong performance for both continuous and discrete data, but:
  \begin{itemize}
    \item Discrete diffusion models typically use fixed-rate transition matrices (e.g., uniform, absorbing).
    \item Fixed matrices limit the expressiveness of latent representations and the design space.
    \item Variational methods (e.g., VAEs) are highly expressive due to adaptive latent encodings.
  \end{itemize}
  \textbf{Goal:} Combine the flexibility of variational inference with the tractability of discrete diffusion.
\end{frame}

\begin{frame}{Limitations of Prior Discrete Diffusion}
  \begin{itemize}
    \item Existing discrete diffusion models use non-learnable, static transition matrices.
    \item This restricts the model’s ability to adapt to complex data.
    \item Only simple matrices (Absorb, Uniform) are computationally feasible.
    \item No adaptive latent variable structure as in VAEs.
  \end{itemize}
  \textbf{Key Question:} Can we design a discrete diffusion model with a learnable, adaptive transition matrix and a variational structure?
\end{frame}

\section{Discrete Markov Bridge Framework}

\begin{frame}{Overview of Discrete Markov Bridge (DMB)}
  \begin{itemize}
    \item DMB is a two-stage, bidirectional learning framework:
      \begin{enumerate}
        \item \textbf{Matrix-learning (Forward Bridge):} Learn a parameterized rate transition matrix to map data to a latent (prior) distribution.
        \item \textbf{Score-learning (Backward Bridge):} Learn a neural network to estimate probability ratios (concrete scores) for reverse generation.
      \end{enumerate}
    \item Integrates variational inference (ELBO) with discrete diffusion.
    \item Enables richer latent representations and improved modeling flexibility.
  \end{itemize}
  \vspace{1em}
  \textbf{Diagram:}
  \[
    \mu \xrightarrow{\text{Matrix-learning}} p_T \xrightarrow{\text{Score-learning}} \hat{\mu}
  \]
\end{frame}

\begin{frame}{Continuous-Time Discrete Markov Chain}
  \textbf{State space:} $X = \{1, 2, \dots, n\}$\\
  \textbf{Markov process:} $\{X(t): t \in \mathbb{R}, X(t) \in X\}$\\
  \textbf{Transition probability:} $p_{t+s|t}(y|x) = \mathbb{P}(X(t+s)=y | X(t)=x)$\\
  \textbf{Marginal:} $p_t(x) = \mathbb{P}(X(t) = x)$\\
  \textbf{Rate matrix:}
  \[
    q_t(x, y) = \frac{d}{ds} p_{t+s|t}(y|x)\bigg|_{s=0}
  \]
  \textbf{Forward Kolmogorov equation:}
  \[
    \frac{d}{dt} p_t = p_t Q(t)
  \]
  where $Q(t)$ is a rate matrix: rows sum to zero, off-diagonals non-negative.
\end{frame}

\begin{frame}{Reversibility and Backward Process}
  \textbf{Reversibility theorem:}\\
  Given the forward equation $\frac{d}{dt} p_t = p_t Q(t)$, there exists a reverse process:
  \[
    \frac{d}{dt} p_{T-t} = p_{T-t} \hat{Q}(T-t)
  \]
  where
  \[
    \hat{Q}(t)_{x,y} = \frac{p_t(y)}{p_t(x)} Q(t)_{y,x}
  \]
  \textbf{Implication:} If we can estimate the probability ratios $\frac{p_t(y)}{p_t(x)}$, we can construct the reverse process and sample from the data distribution.
\end{frame}

\section{Matrix-learning: Adaptive Forward Process}

\begin{frame}{Matrix-learning: Adaptive Transition Matrix}
  \textbf{Key innovation:} Learn a parameterized, diagonalizable rate matrix $Q_\alpha$ with parameters $\alpha$.
  \begin{itemize}
    \item $Q_\alpha$ must have non-negative off-diagonals, rows sum to zero.
    \item Diagonalizability enables efficient computation of matrix exponentials.
    \item The forward process is:
      \[
        \frac{d}{dt} p_t = p_t Q_\alpha(t), \quad p_0 \approx \mu
      \]
    \item The terminal distribution $p_T$ serves as the learned prior.
  \end{itemize}
\end{frame}

\begin{frame}{Matrix-learning Objective}
  \textbf{Objective:} Minimize the KL divergence between the forward conditional and the terminal prior:
  \[
    J_Q = \mathbb{E}_\mu \left[ \mathrm{KL}(p_{T|0;\alpha} \| p_{T;\alpha}) \right]
  \]
  where $p_{T|0;\alpha}$ is the conditional at time $T$ given data, and $p_{T;\alpha}$ is the marginal at $T$.
  \vspace{1em}
  \textbf{Example:} For $X = \{A, B, C\}$, $Q_\alpha$ can be learned so that the process at $T$ yields a desired prior (e.g., uniform or learned).
\end{frame}

\begin{frame}{Diagonalizability and Computational Efficiency}
  \begin{itemize}
    \item Diagonalizable $Q_\alpha$ allows efficient computation of $\exp(\sigma Q_\alpha)$ for transition probabilities.
    \item Structured or sparse parameterizations reduce memory and computation cost.
    \item Enables scaling to large state spaces (e.g., text sequences).
  \end{itemize}
\end{frame}

\section{Score-learning: Reverse Process and Concrete Scores}

\begin{frame}{Score-learning: Neural Score Estimation}
  \textbf{Goal:} Estimate the ratio
  \[
    s_\theta(x, t, y) \approx \frac{p_{t|0}(y|x_0)}{p_{t|0}(x|x_0)}
  \]
  for all $x, y \in X$.
  \begin{itemize}
    \item $s_\theta$ is parameterized by a neural network.
    \item Used to construct the reverse rate matrix:
      \[
        \hat{Q}(t)_{x,y} = s_\theta(x, t, y) Q_\alpha(t)_{y,x}
      \]
    \item Enables the backward process to reconstruct the data distribution from the latent.
  \end{itemize}
\end{frame}

\begin{frame}{Loss Function and Training}
  \textbf{Variational objective:} Continuous-time Evidence Lower Bound (ELBO):
  \[
    \mathrm{ELBO} = \mathbb{E}_{x_0 \sim \mu} \left[ \mathbb{E}_{x_T \sim p_{T|0;\alpha}(\cdot | x_0)} \log p_{0|T;\theta}(x_0 | x_T) - \mathrm{KL}(p_{T|0;\alpha}(\cdot | x_0) \| p_T) \right]
  \]
  \begin{itemize}
    \item $p_{0|T;\theta}$: backward process conditional (decoding).
    \item $p_{T|0;\alpha}$: forward process conditional (encoding).
    \item Jointly optimize $\alpha$ (transition matrix) and $\theta$ (score network) to maximize ELBO.
  \end{itemize}
\end{frame}

\begin{frame}{Theoretical Guarantees}
  \begin{itemize}
    \item DMB provides formal guarantees for reversibility and convergence of the learning process.
    \item Under mild conditions, the learned model converges to the true data distribution as model capacity and data increase.
    \item The ELBO structure ensures principled likelihood-based training and evaluation.
  \end{itemize}
\end{frame}

\section{Computational Considerations}

\begin{frame}{Efficient Computation and Space Complexity}
  \begin{itemize}
    \item Diagonalizable $Q_\alpha$ allows fast computation of matrix exponentials.
    \item Structured/sparse parameterizations (e.g., block-diagonal, low-rank) reduce memory cost.
    \item DMB scales to large vocabularies and sequence lengths, unlike prior models with dense $n \times n$ matrices.
  \end{itemize}
\end{frame}

\section{Empirical Results}

\begin{frame}{Text Modeling: Text8 Dataset}
  \begin{itemize}
    \item DMB achieves an ELBO of 1.38 on Text8, outperforming SEDD and other baselines.
    \item Demonstrates effective modeling of complex natural language distributions.
  \end{itemize}
\end{frame}

\begin{frame}{Image Modeling: CIFAR-10}
  \begin{itemize}
    \item DMB integrated with VQ-VAE achieves results comparable to DDPM on CIFAR-10.
    \item Shows generality across data modalities (text, images).
  \end{itemize}
\end{frame}

\begin{frame}{Ablation and Analysis}
  \begin{itemize}
    \item Both Matrix-learning and Score-learning are essential for optimal performance.
    \item Learned transition matrices adapt to data structure.
    \item Score network accurately reconstructs data from latent.
  \end{itemize}
\end{frame}

\section{Discussion and Conclusion}

\begin{frame}{Advantages and Limitations}
  \textbf{Advantages:}
  \begin{itemize}
    \item Learnable transition matrix enables richer latent representations.
    \item Variational structure provides principled likelihood estimation.
    \item Scalable to large state spaces.
  \end{itemize}
  \textbf{Limitations:}
  \begin{itemize}
    \item Computational overhead for very large vocabularies.
    \item Further exploration needed for other data types (e.g., graphs).
  \end{itemize}
\end{frame}

\begin{frame}{Conclusion}
  \begin{itemize}
    \item Discrete Markov Bridge (DMB) unifies variational inference and discrete diffusion.
    \item Introduces learnable, adaptive transition matrices for flexible modeling.
    \item Achieves state-of-the-art results on text and image tasks.
    \item Opens new avenues for discrete generative modeling research.
  \end{itemize}
  \vspace{1em}
  \textbf{Code:} \url{https://github.com/Henry839/Discrete-Markov-Bridge}
\end{frame}

\begin{frame}{Notation Summary}
  \begin{itemize}
    \item $X = \{1, \ldots, n\}$: Discrete state space
    \item $\mu$: Data distribution
    \item $p_t$: Distribution at time $t$
    \item $Q_\alpha$: Learnable rate matrix
    \item $s_\theta(x, t, y)$: Neural score estimator
    \item $p_{T|0;\alpha}$: Forward conditional
    \item $p_{0|T;\theta}$: Backward conditional
    \item ELBO: Evidence Lower Bound
  \end{itemize}
\end{frame}

\begin{frame}{Example Computation}
  \textbf{Suppose:} $X = \{A, B\}$, $\mu = (0.7, 0.3)$, and
  \[
    Q_\alpha = \begin{bmatrix} -0.5 & 0.5 \\ 0.4 & -0.4 \end{bmatrix}
  \]
  At time $t$, $p_t = (0.6, 0.4)$.\\[1em]
  Reverse matrix:
  \[
    \hat{Q}(t)_{A,B} = \frac{0.4}{0.6} \times 0.4 \approx 0.267
  \]
  \[
    \hat{Q}(t)_{B,A} = \frac{0.6}{0.4} \times 0.5 = 0.75
  \]
  The neural network is trained to estimate these ratios for all $x, y$.
\end{frame}

\begin{frame}{References}
  Discrete Markov Bridge, Hengli Li et al., arXiv:2505.19752 (2025)\\
  \url{https://arxiv.org/abs/2505.19752}
\end{frame}

\end{document}
